\documentclass[11pt,twocolumn]{article}
\usepackage{amssymb,amsmath}
\usepackage{dsfont}
\usepackage{times}
\usepackage[left=1in, right=1in, top=1in, bottom=0.5in, includefoot]{geometry}
\setlength\parindent{0.25in}
\setlength\parskip{1mm}

\title{Gossip Protocol Addresses Jesus in Tights}
\author{Patrick Trinkle \& Mike Corbin\\
Dept. of Computer Science and Electrical Engineering,\\
University of Maryland Baltimore County,\\
Baltimore, MD, 21250\\
\texttt{\{tri1|corbin2\}@umbc.edu}}
\date{December 1st, 2009}

\begin{document}
\twocolumn[
  \begin{@twocolumnfalse}
    \maketitle
    \begin{abstract}
      Seriously awesome stuff about gossip; but we also address it's limitations.  and we use its awesomeness to solve problem X.  Might also, want
 to address its rhobustness.
    \end{abstract}
  \end{@twocolumnfalse}
  ]

\section{Motivation}

The gossip protocol dictates that nodes on the network uniformly
 randomly select other nodes in the network to share information.
  The information shared is small and the sharing doesn't have to
 go both ways.  A node may have information it wishes to send
 off to other nodes in the network; a rumor.  As each node in the
 network learns the rumor it then finds other nodes to tell.  At each
 step in the rumor spreading the state of one or both of the nodes
 changes.  Within $O(\lg N)$ steps all nodes should know the rumor.
  There are many variations in the system which can determine the
 effectiveness of the gossip system.  These variables include: what
 a node does if it pings a node that already knows the rumor; what
 a node does if it has been waiting around--should it ask if there is
 a rumor \cite{Birm2007}.

Erlang provides a stable distributed system and development
 environment.  Therefore it is a good choice for developing applications
 that utilize the gossip protocol \cite{Erlang}.

\section{Background}

Gossip protocol systems try to solve problems as aggregates.  They
 win with small communication requirements and do not assume reliable
 communication \cite{Birm2007}.

\subsection{Erlang as a Distributed System}

Erlang was developed to run real-time code \cite{Erlang}.

\section{Early Experiments}

Before determining which problem to address with the gossip protocol, we
 built several relatively small-scale applications in Erlang.  These applications
 allowed us to further understand the variables involved in a distributed system,
 such as one using the gossip protocol.  An initial application was built
 which simply shared a rumor.  To stop the rumor spreading the nodes would quit
 with certain probability upon contacting a node, which already knew
 the rumor.  From this two applications where quickly developed.  They both
 built a spanning tree with gossip: in the first application the nodes tracked
 their parent; in the second the parents tracked their children.  Nodes discontinued
 the hunt for available children based on a certain probability after
 finding a node that was already in the tree.  A Tic-Tac-Toe learning
 application was also developed wherein the nodes would play tic-tac-toe
 games with other nodes until they learned not to lose.  This doesn't fit the
 traditional gossip model becasue it can take many steps; however, the node
 partners are chosen randomly and it does come to a conclusion.  The nodes
 themselves do transmit a lot of information.  Two variations of the average
 aggregate were programmed.

\subsection{Variables}

Developing a distributed system involves tweaking certain system parameters.  If
 nodes seek rumors during run-time this can change the time required for
 complete knowledge.  Simultaneously, this may not be required.  When a
 node interacts with another node that already knows the rumor the node can quit
 spreading the rumor or initiating interactions.  However, if too many nodes
 quit too quickly then the rumor can die out before all the nodes know.  Therefore,
 there is tweaking in whether or not a node should quit at this point; not just
 the binary notion but there can be a range using random numbers.  If a node
 contacts another node that knows the rumor (assuming they're spreading
 rumors) it can roll a die to determine whether it should quit.  

How quickly they quit; whether ones that don't 
know yet can try to bug their peers for information...

\subsection{Epidemic Rumor Spreading}

Mike, you wrote this part.  Can you talk about how changing variables effected
it?

\subsection{Spanning Tree Building}

I can write about how I had a version that tracked parents, and version that 
tracked children and how they differed in efficieny.  With one you had to
send a message to the root only, but the nodes had to carry more information
with them regarding their children.  But, with the other the nodes would have
to ping their parent node for information.

\subsection{Calculating the Average}

I can talk about this problem.  We did it in class, so I have some notes.  Also,
if you do it with one-way communication then one guy at the end has to time out;
if you do it with pair-wise communication.  Also, if someone messages you with
 a different value you should start spreading again.  I'm fairly certain my 
code doens't do that yet.  But, I can try that out.

\subsection{Tic-Tac-Toe}

Your code was long and transmitted a lot of data around.  That's about all I can
 really say about it.

\section{X}

What problem did we solve and such.

\begin{thebibliography}{10}
\bibitem[Birm2007]{Birm2007}K. Birman, ``The promise, and limitations, of gossip protocols," \emph{ACM SIGOPS Oper. Syst. Rev.}, pp 8-13, 2007
\bibitem[Erlang]{Erlang}http://ftp.sunet.se/pub/lang/erlang/
\end{thebibliography}

\end{document}
