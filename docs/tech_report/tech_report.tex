\documentclass[11pt,twocolumn]{article}
\usepackage{amssymb,amsmath}
\usepackage{dsfont}
\usepackage{times}
\usepackage[left=1in, right=1in, top=1in, bottom=0.5in, includefoot]{geometry}
\setlength\parindent{0.25in}
\setlength\parskip{1mm}

\title{Gossip Protocol Addresses Jesus in Tights}
\author{Patrick Trinkle \& Mike Corbin\\
Dept. of Computer Science and Electrical Engineering,\\
University of Maryland Baltimore County,\\
Baltimore, MD, 21250\\
\texttt{\{tri1|corbin2\}@umbc.edu}}
\date{December 1st, 2009}

\begin{document}
\twocolumn[
  \begin{@twocolumnfalse}
    \maketitle
    \begin{abstract}
      Seriously awesome stuff about gossip; but we also address it's limitations.  and we use its awesomeness to solve problem X.  Might also, want to address its rhobustness.
    \end{abstract}
  \end{@twocolumnfalse}
  ]

\section{Motivation}

Why are we doing this in gossip and erlang.

Also, we can use the paper I found when we talk about the weaknesses of gossip.

\section{Background}

This will include kind of the fundamentals and requirements of gossip protocols.

\subsection{Erlang as a Distributed System}

We'll talk about Erlang.

\section{Early Experiments}

Here we can talk about early code that we wrote and how the variables changed 
and had varying effects.

\subsection{Variables}

These are the things you tweak.  How quickly they quit; whether ones that don't 
know yet can try to bug their peers for information...

\subsection{Epidemic Rumor Spreading}

Mike, you wrote this part.  Can you talk about how changing variables effected
it?

\subsection{Spanning Tree Building}

I can write about how I had a version that tracked parents, and version that 
tracked children and how they differed in efficieny.  With one you had to
send a message to the root only, but the nodes had to carry more information
with them regarding their children.  But, with the other the nodes would have
to ping their parent node for information.

\subsection{Calculating the Average}

I can talk about this problem.  We did it in class, so I have some notes.  Also,
if you do it with one-way communication then one guy at the end has to time out;
if you do it with pair-wise communication.  Also, if someone messages you with
 a different value you should start spreading again.  I'm fairly certain my 
code doens't do that yet.  But, I can try that out.

\subsection{Tic-Tac-Toe}

Your code was long and transmitted a lot of data around.  That's about all I can
 really say about it.

\section{X}

What problem did we solve and such.

\end{document}
